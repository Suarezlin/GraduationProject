\taskmajor{电气工程及其自动化}
\taskpresident{}
\taskapprovaldate{2019-03-05}
\taskschool{电气工程}
\taskclass{电气~512}
\taskauthor{林子牛}
\tasktitle{基于微信小程序及 SpringBoot 的短视频应用开发}
\taskfromyear{2019}
\taskfrommonth{2}
\taskfromday{26}
\tasktoyear{2019}
\tasktomonth{6}
\tasktoday{1}
\taskplace{西安交通大学}

\taskbackground{
    \noindent \uline{\hspace{2em} SpringBoot 是一个约定大配置的新型 Java Web 框架,其设计目的是简化 Spring 应用开发以及减少不必要的配置。SpringBoot 框架在开发微服务应用方面有着非常大的潜力。\hfill}
    
    %\uline{A \hfill}

    \noindent \uline{\hspace{2em} 微信小程序是由腾讯开发的新兴的应用分发形式。主要依托于微信,具有开发周期短、应用分发方便以及跨平台等优势。\hfill}

    \noindent \uline{\hspace{2em} 本毕业设计拟利用 SpringBoot 和微信小程序开发短视频应用程序,其中 SpringBoot 为服务端,微信小程序为客户端。学生在通过此次毕业设计,学习和掌握互联网应用的开发流程、微服务的实践与部署以及了解互联网的体系架构,对于培养学生专业能力以及实践能力非常有帮助。\hfill}
    
    \noindent \uline{\hfill}
    
    \noindent \uline{\hfill}

%    \uline{超级计算机是计算机中功能最强、运算速度最快、存储容量最大的一类计算机,多用于国家高科技领域和尖端技术研究,是一个国家科研实力的体现,它对国家安全,经济和社会发展具有举足轻重的意义。目前由于成本问题,基于超算的深度学习架构还不够流行,这既限制了科研理论的发展,又没有使超算发挥其应有的作用。}
%
%    \uline{因此,本项目将探索在多CPU集群上的深度行人重识别问题,致力于在项目过程中发现和解决问题,推动理论的发展和实际的应用。}
}

\taskrawmaterial{

	\noindent \uline{1. SpringBoot Reference Guide \hfill}
	
	\noindent \uline{2. Bruce Eckel著,Java 编程思想(第4版),北京:机械工业出版社,2007年 \hfill}
	
	\noindent \uline{\hfill}
	
	\noindent \uline{\hfill}

	%\begin{enumerate}
		%\item \uline{SpringBoot Reference Guide \hfill}
		%\item \uline{Bruce Eckel著,Java 编程思想(第4版),北京:机械工业出版社,2007年 \hfill}
	%\end{enumerate}


%    \uline{在行人重识别领域公认的用于评定一个模型效果的数据集有:VIPeR~\mbox{\cite{gray2007evaluating}}、CUHK01~\mbox{\cite{li2012human}}、CUHK03~\mbox{\cite{li2012human}}、Market-1501~\mbox{\cite{zheng2015scalable}}、DukeMTMC-reID~\mbox{\cite{ristani2016MTMC}}。}
%
%    \uline{毕业设计的资料包括国际上前沿的计算机视觉领域,特别是有关行人重识别问题的期刊、会议论文。~\mbox{\cite{ren2015faster,li2014deepreid,ristani2016MTMC,sun2017beyond,he2017mask,he2016deep,chen2018person}}}
}

\taskmaintask{
	\uline{本毕业设计主要是学习和开发基于SpringBoot的短视频应用。                            
该学生具体的工作内容包括:}
    \begin{enumerate}
        \item \uline{了解和学习微信小程序以及SpringBoot框架的特点及相关技术。}
        \item \uline{采用微信小程序开发应用客户端并上线运行。}
        \item \uline{采用SpringBoot作为服务端框架,使用FFmpeg处理短视频,开发一个短视频应用服务端程序。}
        \item \uline{进一步完成应用调试和测试,验证应用可以可靠地工作。}
        \item \uline{撰写论文并完成约3000字翻译。}
    \end{enumerate}
}

\taskrequirement{
    \begin{enumerate}
        \item \uline{设计的软件功能完备,能够正常在智能手机中运行。}
        \item \uline{论文条理清晰,表述清楚,格式规范。}
        \item \uline{英文翻译正确。}
    \end{enumerate}
}

\taskfinalmaterial{
	\begin{enumerate}
		\item \uline{毕业设计论文一本(A4纸, 1.5万字以上)}
		\item \uline{英文翻译(原文和译文,译文翻译字数在3000字以上)}
		\item \uline{程序源码}
	\end{enumerate}
}

\taskothertaskreference{
	\begin{enumerate}
		\item \uline{Craig Walls著,Spring in Action(Fourth Edition),北京,人民邮电出版社,2016年}
		\item \uline{李华兴等著,Java Web 开发实战经典,北京:清华大学出版社,2010年8月}

	\end{enumerate}
}

\taskteachername{甘永梅}
\taskteacherdate{2019-01-07}
\taskauthorname{}