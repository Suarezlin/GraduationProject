\chapter{结论与展望}\label{sec:conclusion}

本文完成了一个完整的短视频应用系统,并在完成设计的基础上从数据库与视频压制两个方面对应用系统进行了优化。这些优化措施在系统的测试过程中取得了理想的效果,对系统性能的提升也有着较大的帮助。

本文对于系统的优化主要基于两个方面:数据库与视频压制。本文中数据库的优化包含两个部分,分别是数据库参数优化与数据库查询优化。参数优化从数据库系统整体入手,进行数据库设置参数上的优化,提高数据库的整体性能,查询优化从具体查询入手,优化本应用的数据库性能。从测试结果来看数据库优化对于数据库系统性能提升的作用是比较大的。视频压制方面主要采用硬件加速与合理的压制参数进行优化。实验证明使用赢家加速可以明显加快压制过程。

本设计在未来也有许多可以扩展的方面。本文数据库优化只局限于单点数据库优化,将来可以将系统应用到数据库集群上,进行数据库集群优化以及读写分离等优化措施进一步提高系统性能。此外,为了短视频播放的兼容性,本文采用了 h264 编码作为视频压制编码,由于最新的 h265 编码可以提供更大的压缩率,所以未来可以将视频压制编码调节为 h265 即可在更小的码率下提供相同的画质。
